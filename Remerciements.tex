\chapter*{Remerciements}

Je voudrais en premier lieu remercier François-Xavier \textsc{Standaert} et Lilian \textsc{Bossuet} pour avoir accepté d'être les rapporteurs de ce manuscrit, et pour leurs retours très instructifs. 
Je remercie également Vanessa \textsc{Vitse} et Hichem \textsc{Sahbi}, qui en plus d'avoir accepté de faire partie de mon jury, ont participé à mon comité de suivi de thèse ces deux dernières années. 
Leurs retours en cours de route m'ont aidé dans l'orientation et la planification de mes travaux.
Je suis reconnaissant à Annelie \textsc{Heuser} et Benoît \textsc{Gérard} d'avoir également accepté de faire partie de mon jury. 
En outre, j'aimerais les remercier tous les deux pour leur implication indirecte dans cette thèse : Annelie en qualité de \emph{sheperd} pour le papier à \textsc{Ches} 2020; et Benoît pour nos échanges à la conférence \textsc{C\&esar} 2018, ainsi que pour son invitation au séminaire \textsc{SemSecuElec} à Rennes en Janvier 2020, du temps où les présentations orales ne se tenaient pas encore sur Zoom.

Cette thèse n'aurait pas pu voir le jour sans Cécile \textsc{Dumas}, mon encadrante.
Cécile m'a fait confiance dès le début, alors que j'étais presque profane en cryptographie.
C'était un pari risqué, mais sa bienveillance et son encadrement au quotidien m'ont très vite rassuré.
Alors pour toutes nos grandes discussions au cours de cette thèse, pour avoir moult fois relu mes écrits, des fois jusque très tard dans la nuit (ou tôt le matin), et même le jour de ton anniversaire; pour la transmission de tout ton savoir et ton savoir-faire, je t'adresse un immense merci et te témoigne ma reconnaissance éternelle.

De même, cette thèse n'aurait pas eu la même envergure sans Emmanuel \textsc{Prouff}, mon directeur de thèse.
Diriger une thèse à distance me semblait un grand défi \textit{a priori}.
Néanmoins, ces trois années ont démontré que cet obstacle pouvait être très bien surmonté et n'enlevait rien à la qualité de l'encadrement, notamment grâce à ta disponibilité, ta pédagogie, et ta vaste connaissance de la littérature scientifique dans la discipline.
À cela, j'ajouterais une immense gratitude pour la finesse de ta relecture de tous mes travaux.
Tes retours et conseils sur la rédaction/construction d'un papier m'ont donné le plaisir d'écrire ; paradoxe pour le non-littéraire que je suis.

Outre mes deux encadrants, je suis reconnaissant envers deux personnes qui m'ont également servi de mentor au cours de cette thèse.
Tout d'abord Eleonora, qui m'a prodigué ses conseils et avis à chaque étape de la thèse au \textsc{Cesti}.
Cette thèse étant le prolongement de la sienne, la lecture de ses travaux et de son manuscrit ont été des références incontournables pour moi.
Ensuite, Pierre-Alain, pour nos conversations ponctuées de conseils techniques.
Tes travaux et ton expertise dans le domaine de l'apprentissage m'ont souvent apporté une nouvelle perspective lorsque j'étais confronté à des problèmes techniques.

Pour faire une bonne thèse, un bon encadrement est une condition nécessaire, mais pas suffisante.
Il faut aussi un bon environnement de travail.
C'est à mettre au crédit de l'ensemble du \textsc{Cesti}, le laboratoire dans lequel j'ai passé ces trois dernières années, et en premier lieu de Anne, qui en a pris les rênes au moment-même où j'intégrais l'équipe.
En bonne cheffe d'orchestre, elle a su tout mettre en oeuvre pour s'assurer du bien-être des thésards.
Je lui suis très reconnaissant de m'avoir accordé sa confiance à en juger par les nombreuses fois où j'ai été ``désigné volontaire'' pour représenter le laboratoire lors de présentations orales.

Un grand merci aux doctorants du labo, mes compagnons d'infortune qui m'ont rejoint progressivement au fil des années, j'ai nommé Vincent (\aka{} le crypto-trader), Gabriel (\aka{} la machine de Boltzmann), et plus récemment Raphaël.
J'espère que vous vous remettrez de mon déménagement, même si je ne vais qu'au bout du couloir.

Merci à Vincent, Laurent et Thomas du \textsc{Ccrc}\footnote{Club Cycliste des Retraités du \textsc{Cesti}.}, pour toutes nos sorties vélos sur les hauteurs de Chartreuse ou à l'assaut des plus hauts cols des Alpes ; ponctuées de débats dont le niveau n'avait souvent d'égal que le dénivelé parcouru.

Merci à Marie, ange gardien du labo, toujours présente quand on a besoin d'elle, que ce soit pour réaligner des traces, trouver un banc de mesures disponible, ou payer sa tournée à la Nat'.
Merci à Claire et Benoît, pour les soirées arrosées dans votre repaire du Vercors, et pour m'avoir sorti d'une bonne fringale l'été dernier : je vous dois encore quelques biscuits \ldots
Merci à tous les autres membres du \textsc{Cesti}, actuels ou passés : Jean-Christophe, Damien, David, Julien, Guillaume, Elisabeth, Frédéric, Antton, Olivier (\(\times 3\)), Philippe (\(\times 3\)), Roland, Marielle, Nicolas, Véronique, Jessy, Stéphanie, Emrick, Yann, Aurélien, Charles, Dorian (félicitations pour ton bébé).
Je n'oublie pas le \textsc{Lsosp} (parce qu'on les aime quand-même) : je remercie en particulier Antoine, Maxime, Valence ainsi qu'Alexis pour nos échanges sur nos travaux respectifs.
De même, j'adresse mes remerciements à Bruno \textsc{Charrat} et Assia \textsc{Tria} qui, même de loin, ont su suivre l'évolution de mes travaux avec bienveillance; aux assistantes du \textsc{Dsys}, Aurélie, Majda, Sandrine et Virginie, pour leur aide et leur réactivité; au service RH du département avec Franck pour m'avoir grandement aidé à surmonter quelques difficultés administratives lors de l'inscription en thèse, et plus récemment Sophie.

Grâce à certains cités précédemment, j'ai pu élargir mon horizon de travail au delà simplement du \textsc{Cesti}, et je tiens à exprimer ma gratitude à ceux qui ont collaboré avec moi sur différents projets.
Tout d'abord, Damien et Nicolas, du \textsc{Lialp}, avec qui nous avons travaillé -- et continuons -- pour le projet \textsc{Claps}, qui m'a permis de rythmer mon confinement.
Ensuite, Rémi \textsc{Strullu} de l'\acrshort{anssi}, pour sa collaboration sur le projet \acrshort{ascad}-v2, matérialisée par nos nombreux échanges de mails et appels téléphoniques.
J'espère que cela ouvrira de nouveaux horizons de recherche, à nous et aux autres membres de la communauté \acrshort{ml} / \acrshort{sca}.

Je tiens à remercier toutes les personnes avec qui j'ai pu échanger lors des divers séminaires / conférences auxquels (du temps où c'était physiquement possible).
Je pense en particulier à Mathieu \textsc{Carbone}, Élie \textsc{Burzstein}, Rémi \textsc{Audebert}, Yanis \textsc{Linge}, Houssem \textsc{Maghrebi}, et tous les autres que j'ai rencontrés à Amsterdam, Darmstadt ou Gardanne.
Leurs retours m'ont été précieux pour orienter mes premières contributions au domaine.

Merci à ceux que j'ai côtoyés à \textsc{Gem} : Angèle pour m'avoir incité à donner des cours en école de commerce, Mustapha et Barthelemy pour la confiance qu'ils m'ont accordée alors que l'enseignement était une toute nouvelle expérience pour moi, et qu'ils m'ont renouvelée en compagnie d'Alain plus récemment.



Viennent alors des remerciements plus personnels, mais non moins importants dans la réalisation de cette thèse.
Mes amis de prépa à Lyon, en particulier Clémentine et Aurélien qui ont commencé leur thèse au \acrshort{cea} en même temps que moi.
Cela m'a énormément rassuré de pouvoir discuter de nos tracas respectifs sur la thèse.
Courage à Théo, Gabriel, Romain, et plus récemment Bastien qui ont choisi cette voie.

Il est sans doute vexé de ne pas avoir été cité précédemment parmi, mais c'est parce qu'il un mérite remerciement à lui tout seul pour son soutien psychologique et sa finesse qui le caractérisent tant : j'ai nommé Michel.
Mon moral ces dernières années aurait été beaucoup plus erratique sans nos innombrables appels Skype transatlantiques.
Merci à Hugo pour nos escapades à Montréal, Grenoble, et Barcelone, qui m'ont permis de souffler un peu dans la frénésie de la thèse.
Merci à tous les autres membres de \emph{la secte}: Mathilde pour m'avoir fait visiter Darmstadt à l'occasion de \textsc{Cosade}, Tiphaine, Oksana et Alex, Pierre, Anthéa, Léa.
Vous avez tous, à un moment ou à un autre, contribué à mon épanouissement lors de ces deux années à Lyon et celles qui ont suivies, et je vous en suis très reconnaissant.

Merci à mes amis Grenoblois de toujours qui ont contribué à me faire penser à autre chose que le travail : Thibaut, Chloé, Nina, Fabien, Alexis, Hugo, Fanny, Julie, Anaïs, David, Constance, JB, Lætitia.
J'espère qu'on pourra fêter ça très bientôt, comme au bon vieux temps !
Merci à Edo et Miguel, pour les super sorties en montagne, en ski de rando ou à VTT, et qui, malgré les multiples plans foireux dans lesquels je les ai emmenés, me sont restés fidèles.

Enfin, je remercie l'ensemble de ma famille pour son soutien indéfectible de toujours.
Ma grand-mère, pour l'immense soutien logistique tout au long de cet été si particulier à Salon-de-Provence lorsque j'écrivais ce manuscrit, en dépit des entraînements de la patrouille de France, de jour comme de nuit.
Merci à mes grands-parents de Sully-sur-Loire, qui m'ont aussi hébergé plusieurs fois durant cette thèse lorsque j'étais de passage sur Paris.
Je réserve mes tout derniers remerciements -- parce que ce sont les plus importants -- à mes parents Pierre \& Solen pour l'immense soutien tout au long de mes études et plus particulièrement ces derniers mois qui n'étaient pas de tout repos ; ainsi que mes trois petites soeurs, Pauline, Claire \& Anne-Lise pour toutes ces années à me supporter, surtout durant ce confinement.

\epigraph{\textit{À mon grand-père.}}{}
% Pour conclure ces remerciements, j'aimerais dédier cette thèse .
% Tout passionné de mathématiques et de physique qu'il était, et bien qu'il n'ait jamais été doué avec l'informatique, 