%%%%%%%%%%%%%%%%%%%%%%%%%%%%%%%%%%%%%%%%%%%%%%%%%%%%%%%%%%%%%%%%%%%%%%%%%%%%%%%%
%                           SETTINGS PAPIER CHES                               %
%%%%%%%%%%%%%%%%%%%%%%%%%%%%%%%%%%%%%%%%%%%%%%%%%%%%%%%%%%%%%%%%%%%%%%%%%%%%%%%%
To verify the tightness of the bounds, we simulate simple \(\traceLength\)-dimensional leakages from an \(n\)-bit sensitive variable \(\Z\).
The traces are defined such that for every \(t \in \llbracket1, \traceLength\rrbracket\):
\begin{equation}
	\xxx_i[t] =
	\begin{cases}
		u_i + b_i \mbox{, if } t \notin \{t_1, \ldots, t_{\order+1}\} \\
		\hWeight(\z_{t,i}) + b_i \mbox{ otherwise}
	\end{cases}
	\enspace ,
\end{equation}
where \((u_i)_i, (b_i)_i\) and all \((\z_{t,i})_i\) are \gls{iid} draws from the following independent random variables.
Respectively, \(\randVar{U} \sim \mathcal{B}(n, 0.5)\),
\(\randVar{B} \sim \mathcal{N}(0, \sigma^2)\),%
\footnote{
	It is recalled that \(\hWeight\) denotes the Hamming weight function, see \autoref{sec:cpa}.
} 
where  and where \((\z_{1,i}, \ldots, \z_{\order+1,i})\) is a \((\order+1)\)-sharing of \(\z_i\) for the bit-wise addition law.
This example corresponds to a situation where the leakages of the shares are hidden among values having no relation with the target, but have the same marginal \gls{pmf}
Since the \(\z_{t, i}\) are drawn uniformly, \(\hWeight(\z_{t, i})\) follows a binomial marginal \gls{pmf} so they are indistinguishable without prior knowledge. 
Hence the choice of a binomial law for \(\randVar{U}\) when emulating non-informative components.
In order to have an \emph{exhaustive} dataset, every possible combination of the \((\order + 1)\)-sharing has been generated and replicated \(q\) times before adding the noise, where \(q\) is given afterwards in the experiments.
Therefore, it contains \(\card{\trainSet} = q \times 2^{(\order+1)n}\) simulated traces.
Once the data were generated, we trained a \gls{dnn} from the hypothesis class \(\hypoclass\) of the \glspl{mlp} with \(\numLayers=1\) hidden layer made of \(\linSize = 1,000\) neurons.
The training loss is naturally the \gls{nll} loss.%
\footnote{
	Beware that in Pytorch and \textsf{Tensorflow}, the \gls{nll} loss is computed with natural logarithms, whereas here one ought to consider the logarithm in base \(2\).
}
The training lasts, after \(T = 200\) epochs,%
\footnote{
	One epoch refers to the number of iterations needed to process the whole dataset through the \gls{sgd} algorithm.
}
by applying the \gls{sgd} algorithm with a learning rate of \(10^{-3}\).
The trained model is denoted \(\MLmodelSGD{\trainSet}\)

Our simulations comprise three main campaigns:
\paragraph{Experiment 1} (secret-sharing only): 
    in this experiment, we set \(\traceLength = \order+1\) in order to  avoid to consider irrelevant input features. 
	The simulations are done over \(n = 4\) bits, \(\order \in \{0, 1, 2, 3\}\) and \(\sigma \in \{0.01, 0.1, 0.2, 0.4, 0.8, 1.6, 3.2\}\).
	We also generate enough data so that the training set is exhaustive, \ie{}, the number of replicas is \(q = 2,000\). 
	With such a generated dataset, we expect to make the estimation error \eqref{eq:err_est} negligible.
	The gap between the \gls{mi} and the \gls{pi} should therefore only be composed of the optimization error~\eqref{eq:err_opt} and the approximation error~\eqref{eq:err_approx}.
	
\paragraph{Experiment 2} (secret-sharing, with uninformative components): 
    in a second experiment, we have \(\traceLength = 40\), including the uninformative components. 
	Since all components share the same margin law, we recall that they cannot be distinguished without knowing \(\Z\).
	Compared to Experiment 1, we might expect the optimization error to be more important because of the potential difficulty induced by the presence of uninformative components. 
	
\paragraph{Experiment 3} (shuffling, no secret-sharing):
	in a third experiment, we set \(\order = 0\), \(\traceLength \in \{2, 4, 16\}\); in other words, \(\traceLength - 1\) uninformative components are added like in Experiment 2, but this time they are randomly shuffled with the only informative component.
	Note that the shuffling is different for each simulated trace so that one cannot guess in which position the informative leakage lies.
	Therefore, we expect the information perceived by the model to be lower than without shuffling~\cite{veyrat-charvillon_shuffling_2012}.
	Besides, \(\sigma \in \{0.04, 0.2, 0.4, 0.8, 1.6, 3.2\}\) here.



\paragraph{\gls{pi} and \gls{mi} Estimation.}
From those experiments, the \gls{pi} is estimated thanks to a \emph{hold-out} dataset of \(1/5\)-th of the size of the training set size, \ie{} those data are not used by the optimization algorithm.
For the sake of comparison, we estimate the \gls{mi} between the target sensitive \(4\)-bit variable and its simulated leakage model with a Monte Carlo sampling of the leakage \gls{pmf} \(\prob{\XXX \given \Z}\).
The methodology is described in \autoref{sec:monte-carlo}.
The only difficulty is to efficiently compute the precise \gls{pmf} \(h: \sensValue \mapsto \fonction{h}{\sensValue} = \prob{\XXX = \xxx \given \Z = \sensValue}\) given a simulated trace \(\xxx\) in the presence of the counter-measures, \ie{} secret-sharing or shuffling.
We describe hereafter of such computations can be done.

\subparagraph{Secret-sharing.}
We benefit from the technique suggested by Lomné \etal{} at \textsc{Ches} 2014~\cite{lomne_how_2014} that we recall hereafter.
Suppose that one knows the \gls{pmf} of each share for a given leakage trace \(\xxx\), denoted as \(\fonction{h_i}{\sensValue} \eqdef \prob{\XXX = \xxx \given \Z_i = \sensValue}\), applying the total probability formula -- see \autoref{sec:probas} -- leads to:
\begin{equation}
    \fonction{h}{\sensValue} = 
    \sum_{\sensValue_1}\cdots \sum_{\sensValue_{\order}} \fonction{h_0}{\sensValue \plusgf \sensValue_1 \plusgf \cdots \plusgf \sensValue_{\order}}\fonction{h_1}{\sensValue_1} \cdots \fonction{h_{\order}}{\sensValue_{\order}} \enspace .
    \label{eq:conv_masking}
\end{equation}
As described above in this section, we assume for our simulations that the leakage of each share \(\z_{t,i}\) is following a Gaussian law \(\normpdf{\hWeight(\z_{t,i})}{\sigma^2}\) so we explicitly know the functions \(h_i\).
It turns out that the right hand-side can be seen as the following convolution product \((h_0 \conv h_1 \conv \cdots \conv h_\order)(\sensValue)\) in the \gls{vecSpace} \(\gf{2}{}^n\).%
\footnote{
	We inform the unfamiliar reader that \(\gf{2}{n}\) denotes the finite field with \(2^n\) elements, whereas \(\gf{2}{}^n\) denotes the same set, seen as a vector space of dimensionality \(n\), with respect to the field \(\gf{2}{}\).
}
Discrete convolutions in \(\gf{2}{}^n\) can be efficiently computed by using a variant of the \gls{dft} called \gls{wh} transform.
Like with a regular fast Fourier transform, this trick allows to compute \(\prob{\XXX = \xxx \given \Z}\) with a runtime complexity of \(\fonction{\mathcal{O}}{\order \cdot \card{\sensVarSet} \cdot\log{\card{\sensVarSet}}}\) instead of \(\fonction{\mathcal{O}}{\card{\sensVarSet}^\order}\) with a naive computation of \autoref{eq:conv_masking}.

The outcomes of those \gls{mi} estimations are depicted by the orange, blue and pink lines in \autoref{fig:exp_MI_1} and \autoref{fig:exp_MI_2}.

\subparagraph{Shuffling.}
We benefit from the analysis conducted by Veyrat-Charvillon \etal{} at \textsc{Asiacrypt} 2012~\cite{veyrat-charvillon_shuffling_2012}. 
The shuffling counter-measure simulated here assumes that no leakage from the permutation indices occurs, so the conditional \gls{pdf} to compute can be written as follows:
\begin{equation}
	\prob{\XXX = \xxx \given \Z = \sensValue} = \frac{1}{\traceLength}\sum_{j=1}^{\traceLength} \prob{\XXX = \xxx \given \Z_{j} = \sensValue} \enspace ,
	\label{eq:prob_shuffling}
\end{equation}
where \(\Z_j\) denotes the random variable modelizing all the shares \(\z_{j,i}\) for \(i \in \llbracket 1, \card{\trainSet} \rrbracket\).
Since in all the terms of the sum in \autoref{eq:prob_shuffling} except one, \(\XXX\) and \(\Z_j\) are independent, ``it boils down to consider \(\traceLength-1\) leakages out of the \(\traceLength\) as algorithmic noise''~\cite{veyrat-charvillon_shuffling_2012}.