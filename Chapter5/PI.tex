%%%%%%%%%%%%%%%%%%%%%%%%%%%%%%%%%%%%%%%%%%%%%%%%%%%%%%%%%%%%%%%%%%%%%%%%%%%%%%%%
%           THE LINK BETWEEN CROSS ENTROPY AND PERCEIVED INFORMATION           %
%%%%%%%%%%%%%%%%%%%%%%%%%%%%%%%%%%%%%%%%%%%%%%%%%%%%%%%%%%%%%%%%%%%%%%%%%%%%%%%%
This section aims at explaining to what extent the \gls{pi} and the cross entropy introduced in the previous section are linked. 
It is argued here that the \gls{pi} actually equals the cross entropy up to constant factors.
Such a link and the reduction argued in \autoref{sec:loss_func} will allow us to guarantee that minimizing the \gls{nll} loss is a consistent approach for solving the 
Leakage Assessment Problem.
It is recalled that the \gls{pi} has been formally defined in \autoref{sec:leakage_assess}.
We also introduce hereafter the \emph{Empirical Perceived Information}, as given in~\cite{bronchain_leakage_2019}.
\begin{definition}[Empirical Perceived Information~\cite{bronchain_leakage_2019}]
    Let \(\MLmodel: \leakSpace \rightarrow \probSet{\sensVarSet}\).
    The \emph{Empirical Perceived Information}, denoted as \(\estPI{\Z}{\XXX}{\MLmodel}{\trainSet}\), is defined from a profiling set \(\trainSet\) as follows:
    \begin{equation}
        \estPI{\Z}{\XXX}{\MLmodel}{\trainSet} \eqdef
        \entrop{\Z} + \sum_{\sensValue \in \sensVarSet} 
        \prob{\Z = \sensValue} \frac{1}{\card{\trainSet}}
        \sum_{\substack{(\xxx, \z) \in \trainSet \\ \z = \sensValue}}
        \log_2 \MLmodel(\xxx)[\z] \enspace .
    \end{equation}
    \label{def:ePI}
\end{definition}
Informally, the \gls{pi} is defined the same way as the \gls{mi}, but by substituting the \emph{uncertainty} of the true \gls{pmf}, namely \(\log_2 \prob{\Z \given \XXX=\xxx}\), with the uncertainty of the approximating \gls{pmf}, namely \(\log_2 \MLmodel(\XXX)\). 
Surprisingly, this substitution is exactly what defines the cross entropy. 
\begin{proposition}
    \label{prop:PI_eq_gen_loss}
    Let \(\Z\) be a random variable with uniform distribution over \(\sensVarSet\) of cardinal \(\card{\sensVarSet}\).
    Then, the cross entropy and the \gls{nll} loss of a model \(\MLmodel \in \hypoclass\) are respectively linked to the \gls{pi} and its empirical estimation as follows:
    \begin{eqnarray}
        \label{eq:PI_eq_gen_loss}
        \log_2{\card{\sensVarSet}} - \LossFunc[\XXX, \Z]{\MLmodel} & = & 
        \PI{\Z}{\XXX}{\MLmodel} \enspace , \\
        \label{eq:ePI_eq_NLL_loss}
        \log_2{\card{\sensVarSet}} - \LossFunc[\trainSet]{\MLmodel} & = & 
        \estPI{\Z}{\XXX}{\MLmodel}{\trainSet} \enspace .
    \end{eqnarray}
\end{proposition}
\begin{proof}
    The assumption about \(\Z\) implies that \(\entrop{\Z} = \log_2{\card{\sensVarSet}}\).
    Injecting the latter result into the definition of the \gls{pi}, and by using the formula of total probabilities for the expected value we have:
    \begin{eqnarray*}
        \PI{\Z}{\XXX}{\MLmodel} & \eqdef & \entrop{\Z} +
        \sum_{\sensValue \in \sensVarSet} \prob{\Z = \sensValue}
        \esper[\XXX | \Z = \sensValue]{\log_2 \MLmodel(\XXX)[\sensValue]} \enspace , \\
        & = & \log_2{\card{\sensVarSet}} + \esper[\Z]{\esper[\XXX | \Z]{\log_2\MLmodel(\XXX)[\Z]}} \enspace , \\
        & = & \log_2{\card{\sensVarSet}} - \esper[\XXX,\Z]{-\log_2\MLmodel(\XXX)[\Z]} \enspace ,\\
        & = & \log_2{\card{\sensVarSet}} - \LossFunc[\XXX, \Z]{\MLmodel} \enspace .
    \end{eqnarray*}
    The proof for the empirical \gls{pi} follows exactly the same reasoning substituting expected values with averages.%
    \footnote{Actually, \autoref{eq:ePI_eq_NLL_loss} only holds if \(\trainSet\) is \emph{balanced}, \ie{} if the number of traces is the same for each class in \(\trainSet\). 
    This can be assumed without loss of generality, since \(\Z\) is drawn uniformly.}
\end{proof}

\begin{figure}
    \centering
    \tikzset{
     %Define standard arrow tip
     >=stealth',
     %Define style for different line styles
     axis/.style={<->},
     important line/.style={thick},
   }
\begin{tikzpicture}[scale=3]
         \coordinate (O) at (0,0);
         \coordinate (endY) at (0,2.2);
         \coordinate (endX) at (3, 0);
         \draw[important line,->] (O) -- (endY) node[above]{};

         % Ces coordonnées décident les limites de dessin au sein du graphe
         \coordinate (HZ) at (0,2);
         \coordinate (HZX) at (0,0.4);

         \coordinate (rightHZ) at (0.5, 2);
         \coordinate (rightHZX) at (0.5, 0.4);
         \coordinate (MI) at ($0.5*(rightHZ)+0.5*(rightHZX)$);
         \coordinate (PI) at (2.5,2);
         \coordinate (PIlow) at (2.5,0.8);
         \coordinate (PImiddle) at ($0.5*(PI)+0.5*(PIlow)$);
         \coordinate (Xent) at (0,0.8);

         % Bornes sup et inf
         \draw[important line, dashed, ceablue] (PI) -- (HZ) node[left] {\(\entrop{\Z}\)};
         \draw[important line, dashed, ceared] (rightHZX) -- (HZX) node[left] {\(\entrop{\Z|\XXX}\)};

         % MI
         \draw[important line, <->, ceared](rightHZ) -- (rightHZX);
        %  \draw (MI) node[right] {\(\textcolor{ceared}{\MI{\Z}{\XXX}}\)};
         \draw (MI) node[right, text width = 5 cm] {\(\textcolor{ceared}{\MI{\Z}{\XXX}}\geq \frac{f(\beta)}{\textcolor{ceaorange}{\numTracesAttackOpt}}\) \\ Chésirey \etal{}~\cite{chesirey_best_2019}};


         % PI
         \draw[important line, <->, ceadarkgreen] (PI) -- (PIlow);
        %  \draw (PImiddle) node[right, text width = 4cm] {\(\textcolor{ceadarkgreen}{\PI{\Z}{\XXX}{\MLmodel}}\)};
         \draw[important line, dashed, ceadarkgreen] (PIlow) -- (Xent);
         \draw (PImiddle) node[right, text width = 4.75 cm] {\(\textcolor{ceadarkgreen}{\PI{\Z}{\XXX}{\MLmodel}} \leq \textcolor{ceared}{\MI{\Z}{\XXX}}\) \\ Bronchain \etal{}~\cite{bronchain_leakage_2019}};
         \draw[important line, dashed, ceadarkgreen] (PIlow) -- (Xent);
         \draw[important line, dashed, ceadarkgreen] (PIlow) -- (Xent) node[left] {\(\LossFunc[\XXX, \Z]{\MLmodel}\)};
         \node[left = of Xent] {};
\end{tikzpicture}


% \begin{tikzpicture}[scale=1.8]
%     \coordinate (O) at (0,0);
%     \coordinate (endY) at (0,2.2);
%     \coordinate (endX) at (3.3, 0);
%     \draw[important line,<->] (endX) node[right]{\gls{sgd} steps} -- (O) -- (endY) node[above] {\(\textcolor{ceadarkgreen}{\LossFunc{\MLmodel}}\)};

%     % Ces coordonnées décident les limites de dessin au sein du graphe
%     \coordinate (HZ) at (0,2);
%     \coordinate (HZX) at (0,0.4);

%     \coordinate (rightHZ) at (3.2, 2);
%     \coordinate (rightHZX) at (3.2, 0.4);
%     \coordinate (MI) at ($0.5*(rightHZ)+0.5*(rightHZX)$);
%     \coordinate (PI) at (3,2);
%     \coordinate (PIlow) at (3,0.5);
%     \coordinate (PImiddle) at ($0.5*(PI)+0.5*(PIlow)$);
%     \coordinate (Xent) at (0,1);

%     % Bornes sup et inf
%     \draw[important line, dashed, ceablue] (rightHZ) -- (HZ) node[left] {\(\entrop{\Z}\)};
%     \draw[important line, dashed, ceared] (rightHZX) -- (HZX) node[left] {\(\entrop{\Z \given \XXX}\)};

%     % MI
%     \draw[important line, <->, ceared](rightHZ) -- (rightHZX);
%     \draw (MI) node[right] {\(\textcolor{ceared}{\MI{\Z}{\XXX}}\)};

%     % NLL loss
%         \draw[important line, ceadarkgreen] [domain=0:1, samples=100] plot (\x, {0.5 + 1.5*exp(-0.03*\x r)});
%         \draw[important line, ceadarkgreen] [domain=0.99:2, samples=101] plot (\x, {0.5 + 1.5*exp(-0.03*\x r)});
%         \draw[important line, ceadarkgreen] [domain=1.99:3, samples=101] plot (\x, {0.5 + 1.5*exp(-0.03*\x r)});
%         % PI
%         \draw[important line, <->, ceadarkgreen] (PI) -- (PIlow);
%         \draw[ceadarkgreen] (PImiddle) node[left] {\(\PI{\Z}{\XXX}{\MLmodelSGD}\)};
% \end{tikzpicture}
    \caption{Illustration of \autoref{prop:PI_eq_gen_loss}.}
    \label{fig:illustration_prop}
\end{figure}

\autoref{prop:PI_eq_gen_loss}, which is illustrated on \autoref{fig:illustration_prop}, tells us that the \gls{pi} can be expressed as a cross entropy, provided that the entropy of the sensitive variable \(\Z\) is known. 
As already pointed out in~\cite[Thm.~6]{bronchain_leakage_2019}, for all \(\MLmodel \in \hypoclass\) we have \(\PI{\Z}{\XXX}{\MLmodel} \leq \MI{\Z}{\XXX}\).%
\footnote{
    This comes from the fact that the gap between the \gls{mi} and the \gls{pi} can be rephrased as a \gls{kld}-divergence term, which is always non-negative.
}
In other words, computing the cross entropy of any deep learning model enables to get a lower bound of the \gls{mi}. 
This tells nothing about the tightness of such a bound yet.
Hopefully, based on the previous results stated in this section, we now know how to tighten this inequality, as stated by the following proposition.
\begin{proposition}
    \label{prop:cor_sup_PI}
    Let \(\trainSet\) be a profiling set and let \(\MLmodelMLE{\trainSet}\) be the maximum likelihood estimator, namely such that \(\MLmodelMLE{\trainSet} \eqdef \argmin_{\MLmodel \in \hypoclass} \LossFunc[\trainSet]{\MLmodel}\). 
    Then: 
    \begin{enumerate}
        \item \(\MLmodelMLE{\trainSet}\) maximizes the empirical \gls{pi},
        \item  the information perceived by \(\MLmodelMLE{\trainSet}\) converges in probabilities towards the maximum of \gls{pi} over \(\hypoclass\).
    \end{enumerate}
    \begin{eqnarray}
        \label{eq:sup_ePI}
        \estPI{\Z}{\XXX}{\MLmodelMLE{\trainSet}}{\trainSet} 
        & \probConv{\card{\trainSet}} & 
        \max_{\MLmodel \in \hypoclass} \PI{\Z}{\XXX}{\MLmodel} 
        \leq \MI{\Z}{\XXX}\\
        \label{eq:sup_PI}
        \PI{\Z}{\XXX}{\MLmodelMLE{\trainSet}} & \probConv{\card{\trainSet}} &
        \max_{\MLmodel \in \hypoclass} \PI{\Z}{\XXX}{\MLmodel} 
        \leq \MI{\Z}{\XXX}
    \end{eqnarray}
\end{proposition}
Roughly speaking, \autoref{prop:cor_sup_PI} states that  the \gls{nll} loss minimization is asymptotically equivalent to the \gls{pi} maximization mentioned in the Leakage Assessment Problem (\ie{} \autoref{leak_assess}).
\begin{proof}
    Starting from \autoref{eq:PI_eq_gen_loss}, and applying \autoref{eq:consist1} from \autoref{thm:consistency} to get
    \begin{eqnarray*}
        \log_2{\card{\sensVarSet}} - \PI{\Z}{\XXX}{\MLmodelMLE{\trainSet}} \overset{\mbox{\tiny\eqref{eq:PI_eq_gen_loss}}}{=} \LossFunc[\XXX, \Z]{\MLmodelMLE{\trainSet}}
        & \overset{\mbox{\tiny\eqref{eq:consist1}}}{\probConv{\card{\trainSet}}} &
        \min_{\MLmodel \in \hypoclass} \LossFunc[\XXX, \Z]{\MLmodel} \\
        & = & \min_{\MLmodel \in \hypoclass} \left(\log_2{\card{\sensVarSet}} - \PI{\Z}{\XXX}{\MLmodel}\right) \\
        & = & \log_2{\card{\sensVarSet}} - \max_{\MLmodel \in \hypoclass} \PI{\Z}{\XXX}{\MLmodel}
    \end{eqnarray*}
    Hence the result given in \autoref{eq:sup_PI}. 
    The proof for \autoref{eq:sup_ePI} follows exactly the same reasoning, replacing \(\PI{\Z}{\XXX}{\MLmodelMLE{\trainSet}}\) by \(\estPI{\Z}{\XXX}{\MLmodelMLE{\trainSet}}{\trainSet}\), and applying \autoref{eq:consist2} instead of \autoref{eq:consist1}.
\end{proof}

Therefore, on the one hand, we have a theoretically grounded method to address the Leakage Assessment Problem (\ie{} \autoref{leak_assess}) thanks to \autoref{prop:cor_sup_PI}, namely by minimizing the \gls{nll} loss.
On the other hand, since it has been argued in \autoref{sec:leakage_assess} that solving the Leakage Assessment was sound in order to address the SCA Optimization Problem, it follows from \autoref{prop:cor_sup_PI} the main result of this chapter, given hereafter.
\begin{corollary}[Main Result]
    \label{cor:main_result}
    Let \(\MLmodelMLE{\trainSet}\) be the maximum likelihood estimator with respect to the profiling set \(\trainSet\), namely such that \(\MLmodelMLE{\trainSet} \eqdef \argmin_{\MLmodel \in \hypoclass} \LossFunc[\trainSet]{\MLmodel}\).
    Then \(\MLmodelMLE{\trainSet}\) asymptotically minimizes the quantity \(\numTracesAttackEst{\MLmodelMLE{\trainSet}, \beta} \eqdef \frac{f(\beta)}{\PI{\Z}{\XXX}{\MLmodel}}\) when the size of \(\trainSet\) tends towards infinity,%
    \footnote{
        We recall that \(f\) is a known, invertible, strictly increasing function defined by Chésirey \etal{}~\cite{chesirey_best_2019}, and \(\beta \in [0,1]\) is the threshold defined in \autoref{eq:eff_sr}.
    }
    which is an upper-bound of the left-hand side in Inequality~\eqref{eq:chésirey}.
\end{corollary}
\begin{proof}
    By applying \autoref{prop:cor_sup_PI}, we get
    \begin{equation*}
        \numTracesAttackEst{\MLmodelMLE{\trainSet}, \beta} \probConv{\card{\trainSet}} 
        \frac{f(\beta)}{\max_{\MLmodel \in \hypoclass} \PI{\Z}{\XXX}{\MLmodelMLE{\trainSet}}}
        = \min_{\MLmodel \in \hypoclass} \frac{f(\beta)}{\PI{\Z}{\XXX}{\MLmodel}} 
        \geq \frac{f(\beta)}{\MI{\Z}{\XXX}}
    \end{equation*}
\end{proof}

In other words, \autoref{cor:main_result} tells us that minimizing the \gls{nll} loss is sound for the SCA Optimization Problem (\ie{} \autoref{final_task_prof}), in the sense that has been argued in \autoref{sec:leakage_assess}, and that the term \(\numTracesAttackEst{\MLmodelMLE{\trainSet}, \beta}\) might be a good approximation of the lower bound of Inequality~\eqref{eq:chésirey}.
However, this also emphasizes that in the pursuit of estimating \(\numTracesAttackOpt(1, \beta)\) through the \gls{nll} minimization, some weaknesses must be discussed.

First, as recalled in \autoref{sec:leakage_assess}, the higher \(\numTracesAttackOpt(1, \beta)\), the looser Inequality~\eqref{eq:chésirey}.
It is therefore of natural interest to verify to what extent the tightness of the latter inequality holds, in view of estimating \(\numTracesAttackOpt(1, \beta)\) by \(\frac{f(\beta)}{\MI{\Z}{\XXX}}\).
This must be at least empirically verified.
Second, the tightness of Inequality~\eqref{eq:sup_PI} is another possible source of imprecision when one wants to substitute the \gls{mi} with the \gls{pi}.
This will be discussed in the next section, and will eventually be verified through simulations and experiments in \autoref{sec:simus} and \autoref{sec:experiments}.