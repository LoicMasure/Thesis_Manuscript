%%%%%%%%%%%%%%%%%%%%%%%%%%%%%%%%%%%%%%%%%%%%%%%%%%%%%%%%%%%%%%%%%%%%%%%%%%%%%%%%
%								INTRO CHAPTER 5             				   %
%%%%%%%%%%%%%%%%%%%%%%%%%%%%%%%%%%%%%%%%%%%%%%%%%%%%%%%%%%%%%%%%%%%%%%%%%%%%%%%%
The work presented in this chapter aims at grounding the use of \glspl{dnn} in the \gls{sca} context, especially when classical counter-measures like secret-sharing and hiding are involved. 
In this chapter, we investigate to what extent the profiled \gls{sca} optimization problem -- \ie{} \autoref{final_task_prof} -- can be solved through the machine learning framework by carefully choosing the underlying loss function.
This is the main problem addressed in this chapter.

We propose a theoretical study of the \glsfirst{nll} loss in different steps depicted in \autoref{fig:diagram_intro}, by enlightening the fact that such a function is strongly linked to a side-channel information theoretic quantity called \gls{pi}, formally introduced by Renauld \etal{} at \textsc{Eurocrypt} 2011~\cite{renauld_formal_2011}, and recently studied by Bronchain \etal{} at \textsc{Crypto} 2019~\cite{bronchain_leakage_2019}.
As a direct consequence, the \gls{pi} can be straightforwardly computed from the \gls{nll} loss.
More interestingly, this implies that the training phase of a deep learning model, through the minimization of the \gls{nll} loss, is actually equivalent to giving the \gls{pi} estimation which is the closest to the \gls{mi} between the leakage and the target sensitive variable.

In parallel, we benefit from the recent works of Chésirey \etal{} at \textsc{Ches} 2019~\cite{chesirey_best_2019}.
The latter ones provide a lower bound of the optimal efficiency metric \(\numTracesAttackOpt(1, \beta)\) -- defined in \autoref{eq:eff_sr} -- depending on the \gls{mi}.
By combining those two results and by translating them into the \gls{ml} terminology, we show that the \gls{sca} efficiency metric can be accurately estimated without even mounting a key recovery, which justifies the \emph{soundness} of the \gls{dl} approach when the latter one is addressed by training \glspl{dnn} through the minimization of the \gls{nll} loss.

\begin{figure}[h]
    \centering
    \tikzset{
	>=stealth',
	%BASE STYLE
	base/.style = {minimum height=1cm,
		           minimum width=2cm,
		           text centered},
	%ROUNDED STYLE
	rounded/.style = {base,
                      rounded corners,
                      text width = 3cm, node distance = 4cm}
}
\begin{tikzpicture}
	\node (n0)  [rounded, fill=blue!25] {\gls{nll} loss};
    \node (n1)  [rounded, fill=blue!25, right of = n0, node distance = 6cm] {\glsfirst{pi}};
    \node (n3)  [rounded, fill=blue!25, below of = n0] {\(\numTracesAttackOpt(1, \beta)\) \\ \tiny see \autoref{eq:eff_sr}};
	\draw[<->] (n0) -- node[above] {\tiny this chapter} (n1);
	\node (n2)  [rounded, fill=blue!25, right of = n3, node distance = 6cm] {\glsfirst{mi}};
	\draw[<->] (n1) -- node[right] {\tiny \cite{renauld_formal_2011,bronchain_leakage_2019}} (n2);
	\draw[<->] (n2) -- node[below] {\tiny \cite{chesirey_best_2019}} (n3);
    \draw[ceadarkgreen, dashed, <->] (n3) -- (n0);

    % \node[circle, fill=ceadarkgreen!20, above right of = n3, node distance = 3cm] (opt) {\(\MLmodel^\star\)};
    \draw[<->, dashed] (n3) -- node[midway, circle, fill=ceadarkgreen!25] (opt){\(\MLmodel^\star\)} (n1);

    \path[->, dashed] (opt) -- node[right]{\tiny maximizes} (n1);
    \draw[->, dashed] (opt) -- node[above right]{\tiny defines} (n2);
    \path[->, dashed] (opt) -- node[above left]{\tiny defines} (n3);
    % À finir si le temps me le permet...
\end{tikzpicture}
    \caption{Link between the \gls{nll} loss and the efficiency metric in \gls{sca}.}
    \label{fig:diagram_intro}
\end{figure}

As we shall show, the latter result has several direct impacts.
First, the training of \glspl{dnn} with the \gls{nll} loss can be considered as an efficient and effective estimation of the \gls{pi}, and thereby of the \gls{mi} -- known to be complex to accurately estimate in the context of secure implementations~\cite{prouff_theoretical_2009,batina_mutual_2011}.
Secondly, it implies that in an \gls{sca} context, choosing the \gls{nll} loss function to drive the training is sound when it comes to address the profiled \gls{sca} optimization problem.
Thirdly, it enables to quantitatively study the impact of classical \gls{sca} counter-measures on the efficiency of deep learning based \gls{sca} and to formally verify that they stay sound.

\paragraph{Outline.}
The remaining of the chapter is organized as follows. 
\autoref{sec:leakage_assess} proposes another way to tackle the evaluation by introducing the \emph{Leakage Assessment} problem.
\autoref{sec:no_free_lunch} states the soundness of minimizing the \gls{nll} loss since it is nothing but maximizing the \gls{pi}.
A discussion about the tightness of such a lower bound can be found in \autoref{sec:div_term}.
The second part of the chapter is dedicated to the validation of our theoretical results through several simulations in \autoref{sec:simus} and experiments presented in \autoref{sec:experiments}, in the context of implementations secured by secret-sharing, shuffling and de-synchronization.