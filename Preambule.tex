%% Copyright (C) 2014 Dorian Depriester
%% http://blog.dorian-depriester.fr
%%
%% This file may be distributed and/or modified under the conditions
%% of the LaTeX Project Public License, either version 1.3c of this
%% license or (at your option) any later version. The latest version
%% of this license is in:
%%
%%    http://www.latex-project.org/lppl.txt
%%
%% and version 1.3c or later is part of all distributions of LaTeX
%% version 2006/05/20 or later.
%%
%% This work has the LPPL maintenance status `maintained'.
%%
%% The Current Maintainer of this work is Dorian Depriester
%% <contact [at] dorian [-] depriester [dot] fr>.
%%
%% This is Preambule.tex for French PhD Thesis.



%%%%%%%%%%%%%%%%%%%%%%%%%%%%%%%%%%%%%%%%
%           Package List		       %
%%%%%%%%%%%%%%%%%%%%%%%%%%%%%%%%%%%%%%%%


%% Blind text
\usepackage{blindtext}

%%%%%%%%%%%%%%%%%%%%%%%%%%%%%%%%%%%%%%%%%%%%%%%%%%%%%%%%%%%%%%%%%%%%%

%% Fonts and typo    
\usepackage[utf8]{inputenc}		% LaTeX, comprend les accents !
\usepackage[T1]{fontenc}
		
% \usepackage[frenchb]{babel}
\usepackage{lmodern}
\usepackage{ae,aecompl}			% Utilisation des fontes vectorielles modernes
% \usepackage[upright]{fourier}
\usepackage{palatino}



%%%%%%%%%%%%%%%%%%%%%%%%%%%%%%%%%%%%%%%%%%%%%%%%%%%%%%%%%%%%%%%%%%%%%
% General overview of the document
\usepackage{enumerate}
\usepackage{enumitem}
\usepackage[section]{placeins}	% Place un FloatBarrier à chaque nouvelle section
\usepackage{epigraph}
\usepackage[font={small}]{caption}
\usepackage[nohints]{minitoc}		% Mini table des matières, en français
\usepackage[notbib]{tocbibind}		% Ajoute les Tables	des Matières/Figures/Tableaux à la table des matières

%%%%%%%%%%%%%%%%%%%%%%%%%%%%%%%%%%%%%%%%%%%%%%%%%%%%%%%%%%%%%%%%%%%%%
%% Maths                         
\usepackage{amsmath}			% Permet de taper des formules mathématiques
\usepackage{amssymb}			% Permet d'utiliser des symboles mathématiques
\usepackage{amsfonts}			% Permet d'utiliser des polices mathématiques
\usepackage{amsthm}				% Permet d'écrire des théorèmes
\usepackage{nicefrac}			% Fractions 'inline'
\usepackage{stmaryrd}			% Pour les brackets
\usepackage{mathtools}			% Pour la KL div

\newtheorem{theorem}{Theorem}
\newtheorem{definition}{Definition}
\newtheorem{proposition}{Proposition}
\newtheorem{lemma}{Lemma}
\newtheorem{remark}{Remark}
\newtheorem{example}{Example}
\newtheorem{problem}{Problem}
\newtheorem{assum}{Assumption}
\newtheorem{corollary}{Corollary}
\newcommand{\definitionautorefname}{Definition}
\newcommand{\propositionautorefname}{Proposition}
\newcommand{\lemmaautorefname}{Lemma}
\newcommand{\remarkautorefname}{Remark}
\newcommand{\exampleautorefname}{Example}
\newcommand{\problemautorefname}{Problem}
\newcommand{\assumautorefname}{Assumption}
\newcommand{\corollaryautorefname}{Corollary}

% Renames sections with Capital letters
\newcommand{\chapterautorefname}{Chapter}
\newcommand{\sectionautorefname}{Section}
\newcommand{\subsectionautorefname}{Subsection}

%%%%%%%%%%%%%%%%%%%%%%%%%%%%%%%%%%%%%%%%%%%%%%%%%%%%%%%%%%%%%%%%%%%%%
%% Tables & algorithms
\usepackage{multirow}
\usepackage{booktabs}
\usepackage{colortbl}
\usepackage{tabularx}
\usepackage{multirow}
\usepackage{threeparttable}
\usepackage{etoolbox}
	\appto\TPTnoteSettings{\footnotesize}
\usepackage{algorithm,algorithmicx,algpseudocode} % For algorithms
\newcommand{\algorithmautorefname}{Algorithm}

%%%%%%%%%%%%%%%%%%%%%%%%%%%%%%%%%%%%%%%%%%%%%%%%%%%%%%%%%%%%%%%%%%%%%
%% Graphics
\usepackage{wrapfig}                   
\usepackage{graphicx}			% Permet l'inclusion d'images
\usepackage{subcaption}
\usepackage{pdfpages}
% \usepackage{rotating}
\usepackage{pgfplots}
	\usepgfplotslibrary{groupplots,dateplot}
\usepackage{tikz}
	\usetikzlibrary{backgrounds,automata,arrows,shadows,positioning, shapes}
	\usetikzlibrary{decorations.pathreplacing,angles,quotes}
	\pgfplotsset{width=7cm,compat=1.3}
\usepackage{import}

%%%%%%%%%%%%%%%%%%%%%%%%%%%%%%%%%%%%%%%%%%%%%%%%%%%%%%%%%%%%%%%%%%%%%
%% Text shape        
\usepackage{xspace}
\usepackage{textcomp}
\usepackage{array}
\usepackage{hyphenat}

%%%%%%%%%%%%%%%%%%%%%%%%%%%%%%%%%%%%%%%%%%%%%%%%%%%%%%%%%%%%%%%%%%%%%
%% Browing through the document
\usepackage[pdftex,pdfborder={0 0 0},
			colorlinks=true,
			linkcolor=blue,  % Changes hyperlink colors
			citecolor=red,
			pagebackref=true,
			]{hyperref}	% Créera automatiquement les liens internes au PDF
					% Doit être chargé en dernier (Sauf exceptions ci-dessous)
			

%%%%%%%%%%%%%%%%%%%%%%%%%%%%%%%%%%%%%%%%%%%%%%%%%%%%%%%%%%%%%%%%%%%%%
%% Packages qui doivent être chargés APRES hyperref	             
\usepackage[top=2.5cm, bottom=2cm, left=3cm, right=2.5cm,
			headheight=15pt]{geometry}

\usepackage{fancyhdr}			% Entête et pieds de page. Doit être placé APRES geometry
	\pagestyle{fancy}		% Indique que le style de la page sera justement fancy
	\lfoot[\thepage]{} 		% gauche du pied de page
	\cfoot{} 			% milieu du pied de page
	\rfoot[]{\thepage} 		% droite du pied de page
	\fancyhead[RO, LE] {}	
	\setlength{\headheight}{25.5pt}
	
\usepackage[acronym,toc,numberedsection,ucmark]{glossaries}
	\newglossary[nlg]{notation}{not}{ntn}{Notation} % Création d'un type de glossaire 'notation'
	\makeglossaries
	\loadglsentries{Glossary}			% Utilisation d'un fichier externe pour la définition des entrées (Glossaire.tex)	


%%%%%%%%%%%%%%%%%%%%%%%%%%%%%%%%%%%%%%%%%%%%%%%%%%%%%%%%%%%%%%%%%%%%%%%%%%
% CEA colors
\RequirePackage{ceacolors}


% Comments & todos
\usepackage[textsize=tiny]{todonotes}  % disable to erase remaining comments
\setlength{\marginparwidth}{2cm}
\newcommand{\comLoic}[1]{\todo[color=blue!20, fancyline]{[Loïc] #1}} 
