%%%%%%%%%%%%%%%%%%%%%%%%%%%%%%%%%%%%%%%%%%%%%%%%%%%%%%%%%%%%%%%%%%%%%%%%%%%%%%%%
%                                   CONCLUSION CHAPTER 6                       %
%%%%%%%%%%%%%%%%%%%%%%%%%%%%%%%%%%%%%%%%%%%%%%%%%%%%%%%%%%%%%%%%%%%%%%%%%%%%%%%%
So far, this chapter answers two questions likely to help both developers of secure implementations, and evaluators mounting \gls{cnn}-based \gls{sca}.

From a developer's point of view, this chapter has studied the effect of two implementations of a code polymorphism counter-measure against several side channel attack scenarios, covering a wide range of potential attackers.
In a nutshell, code polymorphism as an automated tool, is able to provide a strong protection against threat models considering automated and layman attackers, as the evaluated implementations were secure enough against our first attacker models.
Yet, the implementations evaluated are not sound anymore against stronger attacker models.
The soundness of software hiding counter-measures, if used alone, remains to be demonstrated against state-of-the-art attacks, for example by using other configurations of the code polymorphism toolchain, or by proposing new code transformations.  
All in all, our results underline again, if need be, the necessity to combine the hiding and secret-sharing protection principles in a secured implementation.

From an evaluator's point of view, this work illustrates how to leverage \gls{cnn} architectures to tackle the problem of large-scale side-channel traces, thereby narrowing the gap between \gls{sca} literature and concrete evaluations of secure devices where pattern detection and re-alignment are not always possible.
The idea lies in slight adaptations of the \gls{cnn} architectures already used in \gls{sca}, eventually by exploiting the signal properties of the \gls{sca} traces.
Surprisingly, our results emphasize that, though the use of more complex \gls{cnn} architectures has been shown to be sound to succeed \gls{sca} in the literature, it might not be a necessary condition in an \gls{sca} context.
