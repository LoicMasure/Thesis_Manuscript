\newcommand{\inputGridFromX}{0}
\newcommand{\inputGridFromY}{0}
\newcommand{\inputGridToX}{5}
\newcommand{\inputGridToY}{1}
\newcommand{\paddingTo}{5,1}
\newcommand{\inputUnits}{5}
\newcommand{\dilation}{1}
\newcommand{\outputElevation}{70}
\newcommand{\outputBottomLeft}{0,0}
\newcommand{\outputBottomRight}{1,0}
\newcommand{\outputTopLeft}{0,1}
\newcommand{\outputTopRight}{1,1}
\newcommand{\outputTo}{5,1}
\newcommand{\outputGridFrom}{0,0}
\newcommand{\outputGridTo}{1,1}
\definecolor{blue}{RGB}{38,139,210}
\definecolor{cyan}{RGB}{42,161,152}
\definecolor{base01}{RGB}{88,110,117}
\definecolor{base02}{RGB}{7,54,66}
\definecolor{base03}{RGB}{0,43,54}
\begin{tikzpicture}[scale=.5,every node/.style={minimum size=1cm},on grid]
	\begin{scope}[node/.append style={yslant=0.5,xslant=-0.7},yslant=0.5,xslant=-0.7]
	    % Dessine le padding
	    \draw[step=10mm, base03, dashed, thick] (0,0) grid (\paddingTo);

	    % Dessine l'entrée
	    % \inputUnits
	    % \draw[draw=base03, fill=blue, thick] ({0},{1}) rectangle ({2},{3});
	    \foreach \x in { 0,\number\numexpr 1,...,\number\numexpr \inputUnits-1 } {
            \draw[draw=base03, fill=blue, thick] (\x,0) rectangle (\x+1,1);
        }

        % Met en évidence la zone couverte par le filtre
        \foreach \x in { \inputGridFromX,\number\numexpr \inputGridFromX+\dilation,...,\number\numexpr \inputGridToX-1 } {
            \draw[fill=base02, opacity=0.4] (\x,0) rectangle(\x+1,1);
        }
        \draw[step=10mm, base03, thick] (\inputGridFromX, \inputGridFromY) grid
                                        (\inputGridToX, \inputGridToY);
        \coordinate (BL) at (\inputGridFromX,\inputGridFromY);
        \coordinate (BR) at (\inputGridToX,\inputGridFromY);
        \coordinate (TL) at (\inputGridFromX,\inputGridToY);
        \coordinate (TR) at (\inputGridToX,\inputGridToY);
    \end{scope}

    % % Scope pour la sortie de la première couche
    % \begin{scope}[xshift=-5, yshift=\outputElevation,
    %                 every node/.append style={yslant=0.5,xslant=-0.7},
    %                 yslant=0.5,xslant=-0.7]
    %     \draw[dashed] (BL) -- (\outputBottomLeft) (BR) -- (\outputBottomRight)
    %                   (TL) -- (\outputTopLeft)    (TR) -- (\outputTopRight);
    %     \draw[fill=cyan] (0,0) rectangle (\outputTo);
    %     \draw[step=10mm, base03, thick] (0,0) grid (\outputTo);
    %     \draw[fill=base02, opacity=0.4] (\outputGridFrom) rectangle
    %                                     (\outputGridTo);
    %     \draw[base03, thick] (\outputGridFrom) rectangle (\outputGridTo);

    %     \coordinate (BL) at (\inputGridFromX,\inputGridFromY);
    %     \coordinate (BR) at (\inputGridToX,\inputGridFromY);
    %     \coordinate (TL) at (\inputGridFromX,\inputGridToY);
    %     \coordinate (TR) at (\inputGridToX,\inputGridToY);
    % \end{scope}

    % Scope pour la sortie de la seconde couche
    \renewcommand{\outputTo}{1,1}
    \begin{scope}[xshift=-5, yshift=\outputElevation*2,
                    every node/.append style={yslant=0.5,xslant=-0.7},
                    yslant=0.5,xslant=-0.7]
        \draw[dashed] (BL) -- (\outputBottomLeft) (BR) -- (\outputBottomRight)
                      (TL) -- (\outputTopLeft)    (TR) -- (\outputTopRight);
        \draw[fill=ceared] (0,0) rectangle (\outputTo);
        \draw[step=10mm, base03, thick] (0,0) grid (\outputTo);
        % \draw[fill=base02, opacity=0.4] (\outputGridFrom) rectangle
        %                                 (\outputGridTo);
        % \draw[base03, thick] (\outputGridFrom) rectangle (\outputGridTo);
    \end{scope}
\end{tikzpicture}
