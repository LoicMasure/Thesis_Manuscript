%%%%%%%%%%%%%%%%%%%%%%%%%%%%%%%%%%%%%%%%%%%%%%%%%%%%%%%%%%%%%%%%%%%%%%%%%%%%%%%%
%                       RECALLS IN DISCRETE MATH                               %
%%%%%%%%%%%%%%%%%%%%%%%%%%%%%%%%%%%%%%%%%%%%%%%%%%%%%%%%%%%%%%%%%%%%%%%%%%%%%%%%
\label{sec:disc_math}
In this thesis, \gls{gfpn} denotes the \emph{finite field} of \(p^q\) elements.
For the representation of this field, the specifications of the \gls{aes} consider the set of polynomials with coefficients in \(\zSet_p\), whose addition (denoted by \(\plusgf\)) and multiplication (denoted by \(\multgf\)) are done modulo an irreducible polynomial of degree \(q\).
The parameter \(p\), necessarily prime, is called the \gls{characteristic} of the field.
In this thesis, we will only be interested in the \emph{Rijndael field} \(\gf{2}{8} = \mathbb{Z}_2[X] / P(X)\), where \(P(X) = X^8+X^4+X^3+X+1\), on which all the AES operations are defined.
Its \gls{characteristic} being \(2\), it has two consequences.
First, the polynomial coefficients are binary.
Since any polynomial is fully represented by its coefficients, any element in \gf{2}{8} can be seen as a byte value.
Second, the addition between two polynomials being nothing but the element-wise addition of their coefficients in \(\zSet_2\), the field addition \(\plusgf\) coincides with the bit-wise \verb+xor+ operation between two bytes, and thereby the addition coincides with the subtraction.