%%%%%%%%%%%%%%%%%%%%%%%%%%%%%%%%%%%%%%%%%%%%%%%%%%%%%%%%%%%%%%%%%%%%%%%%%%%%%%%%
%							NOTATIONS OF THE MANUSCRIPT						   %
%%%%%%%%%%%%%%%%%%%%%%%%%%%%%%%%%%%%%%%%%%%%%%%%%%%%%%%%%%%%%%%%%%%%%%%%%%%%%%%%
We use calligraphic letters as \mset{X} to denote sets.
If \mset{X} is \gls{finite}, the number of elements in \(\mset{X}\) \aka{} its cardinality is denoted by \card{\mset{X}}.
We use bold notations \(\vectObs{x}\) to denote vectors of elements from a set \(\mset{X}\).

Throughout this thesis, the finite set \(\sensVarSet = \{\sensValue_1, \ldots, \sensValue_N\}\) will be often considered: it will always denote the possible values for a \emph{sensitive} variable \(\Z\).
We will denote by \(\sensValue\) a generic element of \(\sensVarSet\), in contexts in which specifying its index is unnecessary.

% Vectors
When the vectors' orientation minds, they are understood as column vectors.
The \(i\)-th entry of a vector \vectObs{x} is denoted by \(\vectObs{x}[i]\), while the transposed of a vector \vectObs{x} is denoted as \(\vectObs{x}^\intercal\).
We will use the transposed mark to refer to row vectors \(\vectObs{x}^\intercal\).

In this thesis, \(\zSet_p\) denotes the set of relative integers modulo \(p\), and \(\realSet_+\) denotes the set of non-negative integers.

The symbol \(\eqdef\) denotes an equality by definition.
The range of integers from \(a\) to \(b\) included is denoted by \(\llbracket a, b \rrbracket\).
If \(\mathcal{D}\) is a logical statement, we define the \emph{characteristic} function as:
\begin{equation}
	\charac{\mathcal{D}} = 
	\left\{
	\begin{array}{l}
		1 \enspace \mbox{\gls{iff}} \enspace \mathcal{D} \enspace \mbox{is true} \\
		0 \enspace \mbox{otherwise}
	\end{array}
	\right. \enspace .
\end{equation}
Finally, terms in blue are defined in the glossary at the end of this thesis.